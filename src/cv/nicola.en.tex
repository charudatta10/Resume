\documentclass[fontsize=11pt]{tccv}
\usepackage[english]{babel}
\usepackage{graphicx}
\usepackage{multirow}
 \usepackage{fancybox}
\begin{document}
%%% PERSONAL DETAILS %%%
\fancypage{\setlength{\fboxsep}{10pt}\fbox}{}
\centering{\huge Charudatta Gurudas Korde}\\
Research Scholar(PhD)\\
National Institute of Technology\\ 
Farmagudi, Ponda, Goa\\ 
\section{Contact Detail}
\begin{flushleft} Mobile: 8275381582\\
E-mail: korde.charudatta@gmail.com\\
LinkedIn: linkedin.com/in/charudatta-korde-089857139\\
ORCID: 0000-0003-0055-4997\\
Address: H.No. 1055, Primior Bairo ,
Santacruz, Tiswadi, Goa ,403005 .
\end{flushleft}

\vspace{1em}
%%% RESEARCH INTERESTS %%%
\section{Research interests}
Digital VLSI design, Machine Learning, Deep Neural Network, Field Programmable Gate Arrays.
\vspace{1em}
%%% PUBLICATIONS %%%



%%% ACADEMIC PROJECTS %%%
\section{Academic projects}
%%% PROJECT 1 %%%
%\begin{fromto}
%	\item{Present Research}
%\end{fromto}
\begin{project}
	\item{FPGA-based algorithm implementation.}
	{MATLAB, Spyder, Vivado}
	{To design Generative Adversarial Network(GAN) and its variant using the python-based Deep Neural Network(DNN) library namely Keras. The studies is conducted by varying parameters of GANs to get a stable and robust network. The designed network will be implemented on  FPGA  to reduce power consumption and speed up comparable to GPU.  }\\
\end{project}
\vspace{1em}
\section{Publications}
\begin{publi}
\item{2021} {K. G. Shreeharsha, C. G. Korde, M. H. Vasantha and Y. B. Nithin Kumar, \textquotedblleft Training of Generative Adversarial Networks using Particle Swarm Optimization Algorithm\textquotedblright, \textit{Proceedings in 2021 IEEE International Symposium on Smart Electronic Systems (iSES)}.\qquad }
\item{2019} {C. G. Korde, M. Reddy K., V. M. H. and N. K. Y. B, \textquotedblleft Training of Generative Adversarial Networks with Hybrid Evolutionary Optimization Technique\textquotedblright, \textit{Proceedings in 2019 IEEE 16th India Council International Conference (INDICON)}.\qquad }
\item{2018} {Barve, S Raveendran S, Korde C ,Panigrahi T, Nithin Kumar Y, Vasantha M,  \textquotedblleft FPGA implementation of square and cube architecture using vedic mathematics\textquotedblright, \textit{Proceedings in 2018 IEEE 4th International Symposium on Smart Electronic Systems, iSES 2018}.\qquad }
\item{2017} {  Korde C. G., Chandrasekhar, E. and Shenvi, N., \textquotedblleft Multifractal analysis of ionospheric disturbances triggered by earthquakes\textquotedblright, \textit{Proceedings - 18th Annual conference of the International Association of Mathematical Geosciences (IAMG 2017),}Perth, Australia,September, 2017(POSTER).\qquad }
\item{2015} {Korde C. G., Khedekar V. G., Rane K. P. , Nayak A., Mahaddalkar S., \textquotedblleft Implementation of FPGA Based Pre-Processing Algorithms for Devnagri Script Recognition Systems\textquotedblright, \textit{Proceedings in 2018 IEEE Computer Society Annual Symposium on VLSI (ISVLSI),} pp.~164-169,Goa, India, 2015.\qquad }
\end{publi}

\vspace{2em}
%%% EDUCATIONAL QUALIFICATION %%%
\section{Educational qualifications}
\begin{table}[!h]
\centering
{\renewcommand{\arraystretch}{1.1}
	\resizebox{\columnwidth}{!}{
\begin{tabular}{@{}|c|c|c|@{}}
\hline
\textsc{Year} &\textsc{Degree and Institute} &\textsc{Grade}\\
\hline
\multirow{2}{*}{2018 - Present}    & PhD in VLSI   & \multirow{2}{*}{CGPA: 8.6/10}  \\
& National Institute of Technology, Goa & \\
\hline
\multirow{2}{*}{2015 - 2017}    & Master of Engineering in Microelectronics   & \multirow{2}{*}{CGPA: 8.0/10}  \\
& Goa College of Engineering, Goa. & \\
\hline
\multirow{2}{*}{2011 - 2015}  & Bachelor of Technology in EEE   & \multirow{2}{*}{Percentage: 76\% }  \\
 & Goa College of Engineering, Goa. &\\
\hline
%\multirow{2}{*}{2009 - 2011}  & H.S.S.C. (GOA Board)  & \multirow{2}{*}{Percentage: 71 \%}  \\
% & Santa-Cruz Higher Secondary School, Santacruz, Goa. &\\
%\hline
%\multirow{2}{*}{2004 - 2009}  & S.S.C  (GOA Board) & \multirow{2}{*}{Percentage: 69 \%}  \\
% & Dr.K.B.Hedgewar Highschool, Panaji, Goa. &\\
%\hline
\end{tabular}}}
\end{table}
\vspace{1em}
%%% TECHNICAL SKILLS %%%
\section{Technical skills}
\begin{factlist}
\item{Languages}
     {MATLAB, Verilog, Python, C, Julia, VHDL, Cuda.}
 \item{Hardware Platforms }   
 {Nvidia GPU, Raspberry Pi, Arduino, FPGA (Basys 3, ZedBoard).}
\item{Tools}
     {Quartus, MATLAB, Vivado HLX, Spyder (Anaconda).}
\item{Feilds}
     {Deep Neural Networks, Machine Vision - Image Processing, Evolutionary Algorithms, Fuzzy Logic, Cryptography and Network Security, Neural Networks , Fractals.}     
\end{factlist}
%%% WORKSHOPS %%%
\section{work experience}
\begin{fromto}
\item{september 2019 - march 2020}
\end{fromto}
\begin{project}
	\item{Software Validation Engineer}
	{Quartus 19.3, 19.4, 20.1}
	{I interned with Intel Banglore Bellandur SSR4. I worked on testing and validation of the Quartus tool for Partial Reconfiguration and Heraichial design flows.  }
\end{project}
\section{work}
\begin{skills}\\
\item {Github: https://github.com/charudatta10}
\item {Website: https://deathsta.webs.com/}\\
\end{skills}
% \vspace{-1em}
% \section{Acheivements}
% \begin{skills}
% \item {Secured 80.00\% in 8th National IT Aptitude Test.}
% \item {Secured 50.74 marks and 643/1000 score in GATE-2014 with a rank of 2989 out of 3,76,367 students.}
% \end{skills}
% %\section{Achievements}
% %\begin{skills}
% %\item {Received the prestigious Vishveshvarayya fellowship for the present research at NIT Goa }
% %\item {Got 2188 rank in GATE 2014}
% %\item {Got 100\% in Mathematics (both SSC and Intermediate) and in Engineering Drawing}
% %\end{skills}
% %\vspace{-1em}
% %%% EXTRA CURRICULAR ACTIVITIES %%% 
% \section{Extracurricular activities}
% \begin{skills}
% \item {Obtained a certificate of social service from \textquotedblleft GLOBAL CANCER CONCERN INDIA\textquotedblright for raising resources for the cause of cancer sufferers.}
% \item {Worked as volunteer for Unnat Bharat, Swachh Bharat Campaigns, orientation program for freshers and convocations conducted in NIT Goa}
% \end{skills}
% \vspace{-1em}
% %%% PERSONAL SKILLS %%%
% \section{Personal skills}
% Hardworking, good leadership and presentation skills
% \vspace{-1em}
%%% HOBBIES %%%
%%% PUBLICATIONS %%%
% \section{Working Experience}
% \begin{experience}
% \item{Teaching Assistant}
% {IIT Goa}
% {From January 2018 to May 2018}
% {Conducting Analog Circuits lab}
% \vspace{-1em}
% \end{experience}
% \begin{experience}
% \item{Project Staff}
% {NIT GOA}
% {From September 2016 to December 2016}
% {Worked on 10 Bit Successive approximation (SAR) ADC; project was sponsored by SMDP.}
% \vspace{-0.8em}
% \end{experience}
% \vspace*{-1em}

% %% PROJECT 2 %%%
% \begin{fromto}
% 	\item{M.Tech}
% \end{fromto}\vspace{-0.8em}
% \begin{project}
% 	\item{Asynchronous Hybrid Analog to Digital Converters (ADCs)}
% 	{Cadence - Virtuoso, MATLAB}
% 	{This work tends to use a nonuniform sampling whose sampling instances de-
% 		pends on the amplitude of the sampled input voltage. Input voltages near to
		
% 		most significant bit (MSB) will have a lesser delay than input voltages near least
% 		significant bit (LSB). During the operation of one comparator, other comparators
% 		are disconnected form the power supply. This reduces power consumption but
% 		increases the conversion time by introducing the delay.}
% \end{project}
%%% PROJECT 5 %%%
% \begin{fromto}
% 	\item{B.Tech}
% \end{fromto}\vspace{-0.8em}
% \begin{project}
% 	\item{GSM based design of the industrial automation.}
% 	{Keil}
% 	{This work focuses on facing the current issues of catching fire in indus-
% 		tries/home. It senses smoke, fire, increase in the temperature and gas leakage and initiates the buzzer/alarm and sending the message to the controller us-
% 		ing GSM wherever he/she may be. It also displays a warning message in the industry and can informing the police or fire station in case of an emergency.}
% \end{project}
% \vspace{-2em}
%  \section{Publications}
% \begin{publi}
% \item{2018}{Chetan Kamble, Siddharth R. K., Nithin Kumar Y. B., Vasantha M. H., \textquotedblleft Design of Area-Power-Delay Efficient Square Root Carry Select Adder", \textit{4th IEEE International Symposium on Smart Electronic Systems (iSES),} Hyderabad, India, 2018.\qquad [\textit{accepted}]}
% \item{2018}{Siddharth R. K., Sunil R., Nithin Kumar Y. B., Vasantha M. H., Edoardo Bonizzoni, \textquotedblleft An Asynchronous Analog to Digital Converter for Surveillance Camera Applications", \textit{2018 IEEE Computer Society Annual Symposium on VLSI (ISVLSI),} HongKong, China, 2018.}
% \end{publi}
%\vspace{5em}


\vspace{-1em}

%%% REFERENCES %%%
\section{References}
\vspace{-1em}
\begin{figure}[!h]
    \begin{minipage}{.5\textwidth}
\begin{refer}
\item {Dr. M.H. Vasantha}
      {Associate professor, Dept. of ECE, NIT Goa}
      {Email: vasanthmh@nitgoa.ac.in}
\end{refer}
    \end{minipage}%
    \begin{minipage}{0.5\textwidth}
    \begin{refer}
\item {Dr. Y.B.Nithin Kumar}
      {Associate professor, Dept. of ECE, NIT Goa}
      {Email: nithin.shastri@gmail.com}
      \end{refer}
    \end{minipage}
\end{figure}
\end{document}
