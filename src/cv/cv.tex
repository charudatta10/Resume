\documentclass[fontsize=11pt]{tccv}
\usepackage[english]{babel}
\usepackage{graphicx}
\usepackage{multirow}
 \usepackage{fancybox}
\begin{document}
%-------------------------------------------PERSONAL DETAILS
\fancypage{\setlength{\fboxsep}{10pt}\fbox}{}
\centering{\huge Charudatta Gurudas Korde}\\
Research Scholar(PhD)\\
National Institute of Technology\\ 
Farmagudi, Ponda, Goa\\ 
\section{Contact Detail}
\begin{flushleft} Mobile: 8275381582\\
E-mail: korde.charudatta@gmail.com\\
LinkedIn: linkedin.com/in/charudatta-korde\\
ORCID: 0000-0003-0055-4997\\
Address: H.No. 1055, Primior Bairo ,
Santacruz, Tiswadi, Goa ,403005 .
\end{flushleft}
\vspace{1em}

%-------------------------------------------objective
\section{Career Objective}
Looking for a responsible position as a researcher with a view to utilize and
enhance my research and technical skills in a dynamic, growth oriented and technologically driven organization.
\vspace{1em}

%-------------------------------------------RESEARCH INTERESTS
\section{Research interests}
\begin{skills}
    \item {Digital VLSI design.}
    \item {Machine Learning.}
    \item {Deep Neural Network.}
    \item {Field Programmable Gate Arrays.}
\end{skills}
\vspace{1em}

%-------------------------------------------ACADEMIC PROJECTS
\section{Academic projects}

%------PROJECT 1 
\begin{fromto}
	\item{Present Research}
\end{fromto}
\begin{project}
	\item{FPGA based Generative Adversarial Network implementation.}
	{MATLAB, Spyder, Vivado}
	{To design Generative Adversarial Network(GAN) and its variant using python based Deep Neural Network(DNN) library namely Keras. The studies is conducted by varying parameters of GANs to get stable and robust network. The designed network will be implemented on  FPGA  to reduce power consumption and to get speed up comparable to GPU.  }\\
\end{project}
\vspace{1em}

%------PROJECT 2 
\begin{fromto}
	\item{Master in Engineering}
\end{fromto}
\begin{project}
	\item{Investigation of Seismogenic Precursors Using Ionospheric Total Electron Content Data.}
	{MATLAB}
	{Total Electron Content(TEC) Data was gathered from earth based station near earthquake epicenter. Then gathered data is than filtered and processed with Multi Fractal Detrended Fluctuation Analysis(MFDFA) technique. The singularity spectra obtained from MFDFA is used to distinguish between normal TEC data and variation in TEC data due to earthquake.}\\
\end{project}
\vspace{1em}

%------PROJECT 3 
\vspace{1em}
\begin{fromto}
	\item{Bachelors in Engineering}
\end{fromto}
\begin{project}
	\item{ Handwritten Devanagari text recognition and VHDL implimentation.}
	{MATLAB, Xilinx ISE}
	{Raw scanned copies of hand written manuscripts and documents containing Devanagari Scripts (such as Marathi, Kokani, Hindi) were gathered. This raw files were prepossessed to filter out any noise in file. Which are then segmented horizontally to obtain rows. Similar process is is repeated with vertical segmentation get individual letters. The segmented letters are than fed to hopfeild neural network recognition. The Devanagari optical character recognition system discuused was  implemented in MATLAB and VHDL.}\\
\end{project}	
\vspace{1em}

%-------------------------------------------Publications
\section{Publications}

%------Publication 1
\begin{publi}
\item{2021} {K. G. Shreeharsha, C. G. Korde, M. H. Vasantha and Y. B. Nithin Kumar, \textquotedblleft Training of Generative Adversarial Networks using Particle Swarm Optimization Algorithm\textquotedblright, \textit{Proceedings in 2021 IEEE International Symposium on Smart Electronic Systems (iSES)}.\qquad }
\vspace{1em}

%------Publication 2
\item{2019} {C. G. Korde, M. Reddy K., V. M. H. and N. K. Y. B, \textquotedblleft Training of Generative Adversarial Networks with Hybrid Evolutionary Optimization Technique\textquotedblright, \textit{Proceedings in 2019 IEEE 16th India Council International Conference (INDICON)}.\qquad }
\item{2018} {Barve, S Raveendran S, Korde C ,Panigrahi T, Nithin Kumar Y, Vasantha M,  \textquotedblleft FPGA implementation of square and cube architecture using vedic mathematics\textquotedblright, \textit{Proceedings in 2018 IEEE 4th International Symposium on Smart Electronic Systems, iSES 2018}.\qquad }
\vspace{1em}

%------Publication 3
\item{2017} {  Korde C. G., Chandrasekhar, E. and Shenvi, N., \textquotedblleft Multifractal analysis of ionospheric disturbances triggered by earthquakes\textquotedblright, \textit{Proceedings - 18th Annual conference of the International Association of Mathematical Geosciences (IAMG 2017),}Perth, Australia,September, 2017(POSTER).\qquad }
\vspace{1em}

%------Publication 4
\item{2015} {Korde C. G., Khedekar V. G., Rane K. P. , Nayak A., Mahaddalkar S., \textquotedblleft Implementation of FPGA Based Pre-Processing Algorithms for Devnagri Script Recognition Systems\textquotedblright, \textit{Proceedings in 2018 IEEE Computer Society Annual Symposium on VLSI (ISVLSI),} pp.~164-169,Goa, India, 2015.\qquad }
\end{publi}
\vspace{1em}

%-------------------------------------------EDUCATIONAL QUALIFICATION
\section{Educational qualifications}
\begin{table}[!h]
\centering
{\renewcommand{\arraystretch}{1.1}
	\resizebox{\columnwidth}{!}{
\begin{tabular}{@{}|c|c|c|@{}}
\hline
\textsc{Year} &\textsc{Degree and Institute} &\textsc{Grade}\\
\hline
\multirow{2}{*}{2018 - Present}    & PhD in VLSI   & \multirow{2}{*}{CGPA: 8.6/10}  \\
& National Institute of Technology, Goa & \\
\hline
\multirow{2}{*}{2015 - 2017}    & Master of Engineering in Microelectronics   & \multirow{2}{*}{CGPA: 8.0/10}  \\
& Goa College of Engineering, Goa. & \\
\hline
\multirow{2}{*}{2011 - 2015}  & Bachelor of Technology in EEE   & \multirow{2}{*}{Percentage: 76\% }  \\
 & Goa College of Engineering, Goa. &\\
\hline
\multirow{2}{*}{2009 - 2011}  & H.S.S.C. (GOA Board)  & \multirow{2}{*}{Percentage: 71 \%}  \\
 & Santa-Cruz Higher Secondary School, Santacruz, Goa. &\\
\hline
\multirow{2}{*}{2004 - 2009}  & S.S.C  (GOA Board) & \multirow{2}{*}{Percentage: 69 \%}  \\
 & Dr.K.B.Hedgewar Highschool, Panaji, Goa. &\\
\hline
\end{tabular}}}
\end{table}
\vspace{1em}

%-------------------------------------------TECHNICAL SKILLS
\section{Technical skills}
\begin{factlist}
\item{Languages}
     {MATLAB, Verilog, Python, C, Julia, VHDL, Cuda.}
 \item{Hardware Platforms }   
 {Nvidia GPU, Raspberry Pi, Arduino, FPGA (Basys 3, ZedBoard).}
\item{Tools}
     {Quartus, MATLAB, Vivado HLX, Spyder (Anaconda).}
\item{Feilds}
     {Deep Neural Networks, Machine Vision - Image Processing, Evolutionary Algorithms, Fuzzy Logic, Cryptography and Network Security, Neural Networks , Fractals.}     
\end{factlist}
\vspace{1em}

%-------------------------------------------work experience
\section{work experience}
\begin{fromto}
\item{september 2019 - march 2020}
\end{fromto}
\begin{project}
	\item{Software Validation Engineer}
	{Quartus 19.3, 19.4, 20.1}
	{I interned with Intel Banglore Bellandur SSR4. I worked on testing and validation of the Quartus tool for Partial Reconfiguration and Heraichial design flows.  }
\end{project}
\vspace{1em}

%-------------------------------------------Social
\section{work}
\begin{skills}\\
\item {Github: https://github.com/charudatta10}\\
%\item {Website: https://deathsta.webs.com/}\\
\item {Dev.to: https://dev.to/charudatta10}\\
\item {GitHub Pages: https://charudatta10.github.io/myblog/}\\
%\item{Gitlab: https://gitlab.com/152109007c}\\ 
%\item{gitea: https://github.com/charudatta10}\\  
%\item {Upwork: https://github.com/charudatta10}\\
%\item {Freelancer: https://github.com/charudatta10}\\
\end{skills}
\vspace{1em}

%-------------------------------------------Social
\section{Social}
\begin{skills}
\item{Discord: https://discord.gg/6SjyKxVE}\\
%\item{mastodon: https://mastodon.social/@ryunabi326}\\ 
\item{Reddit: https://www.reddit.com/user/ryunabi/}\\ 
\item {Pinterest: https://in.pinterest.com/charudattakorde/}\\
\item {Instagram: https://www.instagram.com/ryunabi326/}\\
\item {YouTube: https://www.youtube.com/channel/UC6rsnLlOAfK6VU-v3kUYqhg}\\
\item {twitch: https://www.twitch.tv/ryunabi326}\\
%\item{Telegram: 8275381582}\\
%\item{WhatsApp: 8275381582}\\
%\item{Signal: 8275381582}\\
%\item{Matrix: https://matrix.to/\#/\#discuss10:matrix.org}\\
%\item{Gravatar: https://gravatar.com/152109007c}\\
%\item{Facebook: https://www.facebook.com/korde.charudatta/}\\ 
%\item{X: https://twitter.com/kordecharudatta}\\
%\item{stackoverflow: https://stackoverflow.com/users/24364782/charudatta-korde}\\
%\item{quora: https://www.quora.com/profile/Charudatta-Korde}\\
\end{skills}
\vspace{1em}

%-------------------------------------------workshop
\section{Workshop}
\begin{skills}
\item {Optical Flare National Workshop on Optical Imaging and Sensing, Optical Communications and Networking, Display Technology, Nano, Opto and Bio Electronics, Remote Sensing and LASERS and their Applications, conducted at NIT Goa, on the 16th and 17th of March, 2013. }
\item {A Workshop on Industrial Automation - Programmable Logic Controller (PLC) \& Supervisory Control and Data Acquisition (SCADA). The workshop gave introduction to PLC design and design of Milk packging plant using  SCADA.  conducted at GEC Goa, from  15th to 18th of Aprill, 2014. }
\item {DSP System Design workshop which was conducted by GEC Goa. Which provided introduction to industry’s next generation model based programming using MATLAB-SIMULINK interface. It also introduced attendee to new developer environment called Vissim. The workshop was  held from 12th to 14th of March, 2015. }
\end{skills}\\
\vspace{8em}

%-------------------------------------------Acheivements
%\section{Acheivements}
%\begin{skills}
%\item {}
%\end{skills}
%\vspace{1em}

%-------------------------------------------EXTRA CURRICULAR ACTIVITIES
\section{EXTRA CURRICULAR ACTIVITIES}
\begin{skills}
\item {Priest.}
\item{Freelancer.}
\item{Investor.}
\item{Learning.}
\end{skills}
\vspace{1em}

%-------------------------------------------PERSONAL SKILLS 
\section{PERSONAL SKILLS}
\begin{skills}
\item {Problem-Solving Abilities.}
\item {Adaptability and Flexibility.}
\item {Creativity and Innovation.}
\item {Emotional Intelligence.}
\item {Teamwork and Collaboration.}
\item {Leadership Skills.}
\item {Critical Thinking.}
\end{skills}
\vspace{1em}

%-------------------------------------------Professional Affiliations
\section{Professional Affiliations}
\begin{skills}
\item {Institute of Electrical and Electronics Engineers (Student Member since 2015).}
\item {IEEE Computational Intelligence Society (2019 - 2020).}
\item {United Nations Children's Fund India (Member since 2016).}
\end{skills}
\vspace{1em}

%-------------------------------------------Professional Affiliations
\section{Hobbies}
\begin{skills}
\item {Coding.}
\item {Gardening.}
\item {Reading.}
\item {Traveling.}
\item {Binge Watch.} 
\item {Playing Chess.}
\item {Cooking.}
\end{skills}
\vspace{1em}

%-------------------------------------------Professional Affiliations
\section{References}
\vspace{-1em}
\begin{figure}[!h]
    \begin{minipage}{.5\textwidth}
\begin{refer}
\item {Dr. M.H. Vasantha}
      {Associate professor, Dept. of ECE, NIT Goa}
      {Email: vasanthmh@nitgoa.ac.in}
\end{refer}
    \end{minipage}%
    \begin{minipage}{0.5\textwidth}
    \begin{refer}
\item {Dr. Y.B.Nithin Kumar}
      {Associate professor, Dept. of ECE, NIT Goa}
      {Email: nithin.shastri@gmail.com}
      \end{refer}
    \end{minipage}
\end{figure}
\end{document}
